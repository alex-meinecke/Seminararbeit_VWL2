\chapter{Fazit}

Grundsätzlich ist ein Emissionshandelssystem ein gutes Instrument, um die Volkswirtschaften klimaneutral zu machen. 
Das Kernkonzept 'Cap and Trade' aus der Theorie finden meistens auch Anwendung, obwohl sie manchmal direkt (Downstream) und manchmal indirekt (Upstream) an die Verbraucher weitergegeben werden können. 

Dennoch gibt es zwei Probleme, die in der Praxis auftreten. 

Ein zentrales Problem ist, dass es keine weltweiten CO2-Preis gibt. 
So können Hersteller die in Volkswirtschaften produzieren, die keinen oder nur einen geringen CO2-Preis haben, einen Wettbewerbsvorteil erlangen.
Das Beispiel des EU ETS hat aber gezeigt, dass das Importierte CO2 nachträglich über Werkzeuge wie den 'Carbon Border Adjustment Mechanism' (CBAM) besteuert werden kann.

Das andere ist, dass Regierungen ihre Zertifikat zu günstig herausgeben und so den Kohlenstoffmarkt überschwemmen. 
Dies mag kurzfristig die Konjunktur ankurbeln, langfristig wird aber die Volkswirtschaft nicht klimaneutral und wird so durch Importe in andere Staaten, die ein Werkzeug wie CBAM anwenden, einen Wettbewerbsnachteil erlangen.