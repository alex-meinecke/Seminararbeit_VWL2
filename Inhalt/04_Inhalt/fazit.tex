\chapter{Fazit}

Emissionshandelssysteme sind als marktwirtschaftliches Instrument gegen den Klimawandel einzigartig aufgrund ihrer ökologischen Treffsicherheit (vgl. \cite[S. 182]{hubert.2020}).
Dies ist auf den sinkenden Emissions-Cap einer Volkswirtschaft zurückzuführen, mit dem die Menge der ausgestoßenen Treibhausgase kontrolliert werden kann.
Voraussetzung dafür ist, dass der Cap nicht zu hoch angesetzt wird, wie am Negativbeispiel der Zertifikatsmenge des EU ETS nach der Finanzkrise gesehen werden kann (vgl. \cite{eu3.2023}).

Ein weiterer Vorteil ist die ökonomische Effizienz. Unternehmen und Verbraucher haben selbst die Wahl, wie sie ihre Emissionen reduzieren.
Sie können selbst entscheiden, wo sie Emissionen vermeiden wollen und wo nicht. Langfristig kann kein Verbraucher das Problem des Treibhausgasausstoßes ignorieren, da sonst die Kosten für seine Emissionen zu hoch werden.
So gibt es Anreize für den privaten Sektor, Geld in die Forschung und Entwicklung klimafreundlicherer Technologien zu investieren (vgl. \cite[S. 183]{hubert.2020}).
Dennoch besteht das Problem, dass es keinen weltweit einheitlichen $CO_2$-Preis gibt und so Hersteller, die in Volkswirtschaften produzieren, die keinen oder nur einen geringen $CO_2$-Preis haben, einen Wettbewerbsvorteil erlangen können.
Das Beispiel des EU ETS hat jedoch gezeigt, dass importierte Treibhausgase nachträglich über Werkzeuge wie den 'Carbon Border Adjustment Mechanism' (CBAM) erfasst werden können und so der $CO_2$-Preis nachträglich auf alle importierten Güter angewendet werden kann (vgl. \cite{ub.2023}).

Bei der Praktikabilität stellt sich die Frage, wie die Zertifikate am besten ausgegeben werden sollen. Der EU ETS hat gezeigt, dass sich ein Downstream-System besonders für große Verursacher in der Industrie eignet.
Die Verteilung durch Auktionen hat sich hier besonders bewährt. Um kleinere Verursacher zu erreichen, können Upstream-Systeme wie das nEHS eingesetzt werden, indem die Verursacher indirekt für ihre Emissionen zahlen müssen.

Alles in allem ist ein Emissionshandelssystem ein gutes Instrument, um marktwirtschaftliche Volkswirtschaften zur Klimaneutralität zu bewegen.