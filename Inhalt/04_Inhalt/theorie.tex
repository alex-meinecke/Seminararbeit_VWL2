\chapter{Theorie}

Das Recht, Treibhausgase zu emittieren war lange Zeit um sonst. 
Wenn für die Produktion eines Gutes Treibhausgase anfielen, wurde dies nicht bei der Preisfindung am Markt zwischen Anbieter und Nachfrager berücksichtig. 
Dabei kostet das Emittieren von Treibhausgasen viel Geld und so wenn es nicht berücksichtig wird in Zukunft auch Wohlfahrt \cite[S. 25]{rabe.2018} \cite{ub4.2023}. 
Diese Kosten können z.B. in Form von Schäden durch Naturkatastrophen, Ernteausfälle oder durch den Anstieg des Meeresspiegels entstehen. 
Das Umweltbundesamt geht bei einer Geleichgwichtung klimawandelverursachter Wohlfahrtseinbußen heutiger und zukünfitger Generationen von Kosten in Höhe von 809 Euro pro Tonne CO2 aus \cite{ub4.2023}.
Um die Mehrkosten von Treibhausgasen zu berücksichtigen, müssen diese Kosten in die Preisfindung am Markt internalisiert werden.

In einem Emissionenshandelssystem werden die Rechte, Treibhausgase zu emittieren, als Zertifikate dargestellt \cite[S. 27]{rabe.2018}. 
Ein Zertifikat berechtigt den Besitzer, eine Tonne CO2-Äquvalente (CO2e) zu emittieren. 
Es gibt neben CO2 verschiedene Treibhausgase, die unterschiedlich stark zum Klimawandel beitragen, da CO2 wird dies als Referenzgröße verwendet. 
Es wird hier auch vom 'Global Warming Potential' gesprochen. So hat bspw. Methan ein 28-fach höheres Global Warming Potential als CO2 \cite{ub.2023}.

\section{Cap and Trade}

Aus den Klimaschutzzielen einer Regierung wird eine Obergrenze für die Emissionenzertifikate bis zu einem bestimmten Zeitraum festgelegt, also der Zeitpunkt bis die Volkswirtschaft klimaneutral sein soll.
Diese Obergrenze wird auch Cap genannt. Dieser Cap wird nun jährlich meistens über eine Aktution an die Verbraucher verteilt. 
Anfangs werden noch verhältnismäßig viele Zertifikate ausgegeben, die Menge wird aber jährlich reduziert. So steigt der Preis der Zertifikate an. Nach der Ausgabe der Zertifikate können diese frei gehandelt werden (Trade). Verbraucher, die mehr Zertifikate benötigen, als sie erhalten haben, müssen diese von anderen Unternehmen kaufen. 
Andersrum können Verbraucher Zertifikate verkaufen, die sie nicht benötigen. So entsteht eine Lenkungswirkung Verbraucher dazu zu bringen, weniger Treibhausgase zu emittieren und auf klimaneutralere Technologien umzusetzen. 
Verbraucher, die CO2e emittieren für das sie keine Zertifikate besitzen, müssen hohe Strafzahlungen leisten. Verbraucher, die sich nicht rechtzeitig reformieren, sind auf Grund der steigenden CO2e-Preise weniger wettbewerbsfähig und werden vom Markt verdrängt.

