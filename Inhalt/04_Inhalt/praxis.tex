\chapter{Praxis}

Aktuell sind weltweit 28 Emissionshandelssysteme aktiv, die ca. 17 \% der weltweiten Treibhausgasemissionen abdecken \cite[S. 7]{icap.2023}. 
Dieses Kapitel beschäftigt sich mit den Emissionshandelssystemen die aktuell u.a. die deutsche Volkswirtschaft decken. 
Stand Ende 2023 sind in Deutschland zwei Emissionshandelssysteme aktiv. 
Zum einen das EU-Emissionshandelssystem (EU ETS) geregelt durch das 'EU ETS legislative framework' \cite{eu.2023} und zum anderen das deutsche nationale Emissionshandelssystem (nEHS), das durch das 'Brennstoffemissionshandelsgesetz' (BEHG) geregelt wird \cite{dehst.2023}.
An diesen Emissionenshandelssystemen werden Best Practices aber auch Probleme und mögliche Lösungsansätze aufgezeigt.

\section{EU-Emissionshandelssystem (EU ETS)}

Das ETS ist das weltweit größte Emissionshandelssystem. Es wurde 2005 eingeführt und deckt ca. 40\% der Treibhausgasemissionen der EU ab. Das ETS gilt für den europäischen Binnenmarkt sowie für die Staaten Island, Lichtenstein und Norwegen. 
Es zielt auf direkte Emissionen aus der produzierenden Industrie, dem Energiesektor und der Luftfahrt ab. Ab 2024 sind auch Emissionen aus der Schiffahrt vom ETS gedeckt \cite{eu.2023}.
Bei dem ETS handelt es sich um einen s.g. Downstream Emissionshandel \cite{dehst.2023}. \todo{Abb. für Upstream and Downstream Emissionshandel}
Das bedeutet, dass Emittenten die Berechtigung für ihre eigenen Emissionen selbst erwerben und so von sich aus einen Anreiz haben, ihre Emissionen zu reduzieren. Ziel des ETS ist es, die Klimavorhaben der EU zu erreichen. 
Das bedeutet mittelfristig die Treibhausgasemissionen bis 2030 im Vergleich zu 1990 um 55\% zu reduzieren und langfristig bis 2050 klimaneutral zu werden \cite{eu.2023}. 

Kern des ETS ist das für Emissionshandelssystemen klassische "Cap and Trade" \cite{eu.2023}. Genau wie in der Theorie werden die Zertifikate am Anfang von den Staaten zu einem Festpreis ausgegeben und nachträglich an Zertifikatsbörsen wie dem European Energy Exchange (EEX) gehandelt. 
Dennoch gibt es hier einige Besonderheiten. So erhalten Emittenten aus der Industrie einen Teil ihrer Berechtigungen von den Staaten kostenlos. 
Grund dafür ist das s.g. 'Carbon Leakage' (dt. Kohlenstoffleck) \cite{eu2.2023}.
Dies beschreibt den Effekt, wenn Unternehmen teibhausgasintensive Produktionen ins Ausland verlagern, um so die Kosten des EU ETS zu umgehen und wettbewerbsfähig zu bleiben. 
Um die Wettbewerbsfähigkeit der Industrie im europäischen Binnenmarkt zu sichern, werden deshalb bisher ein Teil der Berechtigungen kostenlos vergeben. 
So kann wenigsten sichergestellt werden, dass die Emissionen noch durch den 'cap' gedeckelt werden, auch wenn den Unternehmen die Kosten erspart bleiben.

Eine weitere Besonderheit des ETS ist die 'Market Stability Reserve' (MSR) (dt. Markstabilitätsreserve) \cite{eu3.2023}. Sie wurde Anfang 2019 eingeführt und soll auf der einen Seite den Zertifikatsüberschuss resultierend aus der Finanzkrise abzubauen. 
Damals wurde ein Überschuss an Zertifikaten von der EU in Umlauf gebracht, um die europäische Industrie entlasten. 
Doch dieser Überschuss hat seit 2009 die Preise einbrechen lassen und so die Anreize für Unternehmen, ihre Emissionen zu reduzieren, geschwächt. Nach der Einführung der MSR hat sich der Preis allein in einem Jahr verdreifacht \todo{Grafik Preis EU ETS}.
Auf der anderen Seite soll die MSR das EU ETS auch gegen externe Schocks absichern \cite{eu3.2023}. 
So werden jährlich abhangig von den 'Total Number of Allowances in Circulation' (TNAC) (dt. Anzahl der Zertifikaten im Umlauf) entweder Zertifikate aus dem Markt in die MSR als Reserve überführt (TNAC über 833 Mio.) oder wieder ausgegeben (TNAC unter 400 Mio.) \cite[S. 7]{icap2.2023}. 
Um den o.g. Überschuss abzubauen, wurden von 2019 bis 2023 jährlich 24\% der Zertifikate, die im Umlauf waren, in die MSR überführt. 
Danach soll die Einlagerung auf max. 12\% reduziert werden.
Damit die MSR nicht so groß wird und es so zu einem Preiseinbruch kommen kann, kann die MSR maximal die Größe der ausgeben Zertifikate aus dem Vorjahr haben. 
Alle Zertifikate die darüber hinaus in die MSR überführt werden sollten, werden gelöscht \cite{eu3.2023}. 

- Phasen des ETS

\section{Nationales Emissionshandelssystem (nEHS)}

Seit 2021 hat Deutschland ein eigenes Emissionshandelssystem, das nationale Emissionshandelssystem (nEHS) \cite{dehst.2023}. Es soll den EU ETS ergänzen, indem es auch die Emissionen aus den Sektoren Verkehr und Wärme abdeckt, die überwiegend von Privatpersonen erzeut werden. 
Damit nicht alle Privatpersonen direkt am nEHS teilnehmen müssen, wurde das nEHS als Upstream Emissionshandelssystem konzipiert. 
Das bedeutet, dass die Berechtigungen für die Emissionen von den Unternehmen erworben werden, die Brennstoffe, durch welche später die Treibhausgasemissionen verursacht werden, in den Verkehr bringen (s.g. BEHG-Verantwortliche). 
Die Kosten werden dann an die Verbraucher weitergegeben, der dadurch einen Anreiz hat, emissionsarmere Technologien zu nutzen. Beispiel: Eine Raffinerie verkauft Benzin an eine Tankstelle. 
Bei der Verbrennung von einem Liter Bezin entstehen 2,3 kg CO2. Bei einem Preis von 45 €/t CO2e entstehen so zusätzliche Kosten von 10,35 Cent pro Liter Benzin. 
Die notwendigen Zertifikate dafür werden von dem BEHG-Verantwortlichen (Raffinerie) erworben und über den Händler (Tankstelle) an den Verbraucher weitergegeben.

Das "Cap and Trade" wurde beim nEHS genau wie beim EU ETS von der Theorie abgewandelt \cite{dehst.2023}. Die Obergrenze (Cap) wird von der EU-Klimaschutzverordung vorgegeben. 
In der Einführungsphase bis 2026 kann aber der Cap überschritten werden, da sich hier die Veräußerung von Zertifikaten nach der effiktiven Nachfrage der BEHG-Verantwortlichen richtet.
Die EU-Klimaschutzverordung verpflichtet Deutschland bei einer Überschreitung des Caps, dieses Defifizit auszugleichen. Auch der "Trade-Aspekt" wurde abgewandelt \cite{dehst.2023}. 
So werden bis 2025 nicht in einer Versteigerung sondern zu einem Festpreis verkauft, der jährlich steigt \todo{Abb. mit NEUEN Preisen}. 
Ab 2025 werden die Zertifikate dann in einer Versteigerung ausgegeben. 
Es beleibt aber dennoch ein Mindest- und Höchstpreis bestehen, in dessen Rahmen sich dann auch der Preis jeh nach Nachfrage entwickeln wird.

Wie das EU ETS muss sich auch das nEHS des Problems des 'Carbon Leakage' stellen \cite{dehst2.2023}. Anders als beim EU ETS gibt es hier auch Wettbewerbsnachteile im Vergleich zu anderen EU Staaten, die sich auch im Binnenmarkt befinden. 
So kann es z.B. zu Phänomen wie dem 'Tanktourismus' kommen, bei dem Verbraucher, die in Grenzregionen leben, in andere EU Staaten fahren, um dort günstiger zu tanken. 
Während das EU ETS dieses Problem durch kostenlose Zertifikate für die Industrie löst, bietet die deutsche Emissionshandelsstellen (DEHSt), die Möglichkeit s.g. Beihilfen zu beantragen.

Alle Erträge aus dem nEHS werden in den s.g. Klima- und Transformationsfond der Bundesregierung (KTF) eingezahlt, um damit Projekte gegen den Klimawandel zu finazieren \cite{dehst.2023}.

\section{Erfolge unsere Emissionshandelssysteme}



\section{Probleme und Verbessungmöglichkeiten}
- Reform des ETS1
- Souziale Probleme
- Besseres bekämpfen von Carbon Leakage
- Einführung ETS2 und verbindung mit nEHS




