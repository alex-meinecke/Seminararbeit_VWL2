\chapter{Praxis}

Aktuell sind in Deutschland zwei Emissionshandelssysteme aktiv. Zum einen das EU-Emissionshandelssystem (ETS) geregelt durch das 'EU ETS legislative framework' \cite{eu.2023} und zum anderen das deutsche nationale Emissionshandelssystem (nEHS), das durch das 'Brennstoffemissionshandelsgesetz' (BEHG) geregelt wird \cite{dehst.2023}.

\section{EU-Emissionshandelssystem (ETS)}

Das ETS ist das weltweit größte Emissionshandelssystem. Es wurde 2005 eingeführt und deckt ca. 40\% der Treibhausgasemissionen der EU ab. Das ETS gilt für den europäischen Binnenmarkt sowie für die Staaten Island, Lichtenstein und Norwegen. Es zielt auf direkte Emissionen aus der produzierenden Industrie, dem Energiesektor und der Luftfahrt ab. Ab 2024 sind auch Emissionen aus der Schiffahrt vom ETS gedeckt \cite{eu.2023}.
Bei dem ETS handelt es sich um einen s.g. Downstream Emissionshandel \cite{dehst.2023}. Das bedeutet, dass Emittenten die Berechtigung für ihre Emissionen selbst erwerben. Das bedeutet, dass die Emittenten von sich aus den Anreiz haben, ihre CO2e-Austoß zu senken.  
