\chapter{Einleitung}

Der Klimawandel ist eine der größten Herausforderungen unserer Zeit. Um gegen Phänomen vorzugehen, müssen die Volkswirtschaften klimaneutral werden (vgl. \cite[S. 162]{hubert.2020}).
Doch das Recht, Treibhausgase zu emittieren, war lange Zeit kostenlos. Wenn so für die Produktion eines Gutes Treibhausgase anfielen, wurde dies nicht bei der Preisfindung am Markt zwischen Anbieter und Nachfrager berücksichtigt (vgl. \cite[S. 161]{hubert.2020}).
Dabei entstehen bei der Emission von Treibhausgasen externe Kosten, die sich durch langfristige Wohlfahrtsverluste bemerkbar machen werden (vgl. \cite[S. 25]{rabe.2018}) (vgl. \cite{ub4.2023}).
Diese Kosten können z.B. in Form von Schäden durch Naturkatastrophen, Ernteausfällen oder durch den Anstieg des Meeresspiegels entstehen.
Das Umweltbundesamt geht bei einer Gleichgewichtung klimawandelverursachter Wohlfahrtseinbußen heutiger und zukünftiger Generationen von Kosten in Höhe von 809 Euro pro Tonne $CO_2$ aus (vgl. \cite{ub4.2023}).
Bei verfälschter Preisbildung am Markt wird auch von Marktversagen gesprochen (vgl. \cite[S. 268]{hubert.2019}).
Um die externen Mehrkosten von Treibhausgasen zu berücksichtigen und so die Fehlpreisbildung zu korrigieren, müssen diese in die Preisfindung am Markt internalisiert werden (vgl. \cite[S. 161]{hubert.2020}).

Es gibt verschiedene Lenkungsmöglichkeiten, die Staaten einsetzen können, um ihre Volkswirtschaften klimaneutral zu machen.
Neben einer $CO_2$-Steuer und Verhandlungslösungen nach dem Coase-Theorem gibt es das Konzept der Emissionshandelssysteme (vgl. \cite[S. 161]{hubert.2020}).
In dieser Seminararbeit wird Letzteres genauer in Theorie und Praxis betrachtet.