\chapter{Einleitung}

Der Klimawandel ist eine der größten Herausforderungen unserer Zeit. Um gegen dieses Phänomen vorzugehen, müssen die Volkswirtschaften klimaneutral werden. 
Doch das Recht, Treibhausgase zu emittieren, war lange Zeit um sonst.
Wenn für die Produktion eines Gutes Treibhausgase anfielen, wurde dies nicht bei der Preisfindung am Markt zwischen Anbieter und Nachfrager berücksichtig \cite[S. 161]{hubert.2020}.
Dabei entstehen bei der Emission von Treibhausgasen externe Kosten, die sich durch langfristige Wohlfahrtsverluste bemerkbar machen werden \cite[S. 25]{rabe.2018} \cite{ub4.2023}. 
Diese Kosten können z.B. in Form von Schäden durch Naturkatastrophen, Ernteausfälle oder durch den Anstieg des Meeresspiegels entstehen. 
Das Umweltbundesamt geht bei einer Geleichgewichtung klimawandelverursachter Wohlfahrtseinbußen heutiger und zukünfitger Generationen von Kosten in Höhe von 809 Euro pro Tonne CO2 aus \cite{ub4.2023}.
Bei verfälschten Preisbildung am Markt, wird auch von Marktversagen gesprochen \cite[S. 268]{hubert.2019}.
Um die externen Mehrkosten von Treibhausgasen zu berücksichtigen, müssen diese in die Preisfindung am Markt internalisiert werden \cite[S. 161]{hubert.2020}.

Es gibt verschiedene Lenkungsmöglichkeiten, die Staaten einsetzen können, um ihre Volkswirtschaften klimaneutral zu machen.
Neben einer CO2-Steuer und Verhandlungslösungen nach dem Coase-Theorem, gibt es das Konzept der Emissionshandelssysteme \cite[S. 161]{hubert.2020}. 
In dieser Seminarbeit wird letzteres genauer in Theorie und Praxis betrachet.