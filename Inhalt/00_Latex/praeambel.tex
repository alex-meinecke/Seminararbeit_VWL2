\documentclass[
	fontsize=12pt,           % Leitlinien sprechen von Schriftgröße 12.
	paper=A4,
	twoside=false,
	listof=totoc,            % Tabellen- und Abbildungsverzeichnis ins Inhaltsverzeichnis
	bibliography=totoc,      % Literaturverzeichnis ins Inhaltsverzeichnis aufnehmen
	titlepage,               % Titlepage-Umgebung anstatt \maketitle
	headsepline,             % horizontale Linie unter Kolumnentitel
	abstract,              % Überschrift einschalten, Abstract muss in {abstract}-Umgebung stehen
]{scrreprt}                  % Verwendung von KOMA-Report
\usepackage[utf8]{inputenc}  % UTF8 Encoding einschalten
\usepackage[ngerman]{babel}  % Neue deutsche Rechtschreibung
\usepackage[T1]{fontenc}     % Ausgabe von westeuropäischen Zeichen (auch Umlaute)
\usepackage{microtype}       % Trennung von Wörtern wird besser umgesetzt
\usepackage{lmodern}         % Nicht-gerasterte Schriftarten (bei MikTeX erforderlich)
\usepackage{graphicx}        % Einbinden von Grafiken erlauben
\usepackage{wrapfig}         % Grafiken fließend im Text
\usepackage{setspace}        % Zeilenabstand \singlespacing, \onehalfspaceing, \doublespacing
\usepackage[
	%showframe,                % Ränder anzeigen lassen
	left=2.7cm, right=2.5cm,
	top=2.5cm,  bottom=2.5cm,
	includeheadfoot
]{geometry}                      % Seitenlayout einstellen
\usepackage{scrlayer-scrpage}    % Gestaltung von Fuß- und Kopfzeilen
\usepackage{acronym}             % Abkürzungen, Abkürzungsverzeichnis
\usepackage{titletoc}            % Anpassungen am Inhaltsverzeichnis
\contentsmargin{0.75cm}          % Abstand im Inhaltsverzeichnis zw. Punkt und Seitenzahl
\usepackage[                     % Klickbare Links (enth. auch "nameref", "url" Package)
  hidelinks,                     % Blende die "URL Boxen" aus.
  breaklinks=true                % Breche zu lange URLs am Zeilenende um
]{hyperref}
\usepackage[hypcap=true]{caption}% Anker Anpassung für Referenzen
\urlstyle{same}                  % Aktuelle Schrift auch für URLs
% Anpassung von autoref für Gleichungen (ergänzt runde Klammern) und Algorithm.
% Anstatt "Listing" kann auch z.B. "Code-Ausschnitt" verwendet werden. Dies sollte
% jedoch synchron gehalten werden mit \lstlistingname (siehe weiter unten).
\addto\extrasngerman{%
	\def\equationautorefname~#1\null{Gleichung~(#1)\null}
	\def\lstnumberautorefname{Zeile}
	\def\lstlistingautorefname{Listing}
	\def\algorithmautorefname{Algorithmus}
	% Damit einheitlich "Abschnitt 1.2[.3]" verwendet wird und nicht "Unterabschnitt 1.2.3"
	% \def\subsectionautorefname{Abschnitt}
}

% ---- Abstand verkleinern von der Überschrift 
\renewcommand*{\chapterheadstartvskip}{\vspace*{.5\baselineskip}}

% Hierdurch werden Schusterjungen und Hurenkinder vermieden, d.h. einzelne Wörter
% auf der nächsten Seite oder in einer einzigen Zeile.
% LaTeX kann diese dennoch erzeugen, falls das Layout ansonsten nicht umsetzbar ist.
% Diese Werte sind aber gute Startwerte.
\widowpenalty10000
\clubpenalty10000

% ---- Für das Quellenverzeichnis
\usepackage[
	backend = biber,                % Verweis auf biber
	language = auto,
	style = numeric,                % Nummerierung der Quellen mit Zahlen
	sorting = none,                 % none = Sortierung nach der Erscheinung im Dokument
	sortcites = true,               % Sortiert die Quellen innerhalb eines cite-Befehls
	block = space,                  % Extra Leerzeichen zwischen Blocks
	hyperref = true,                % Links sind klickbar auch in der Quelle
	%backref = true,                % Referenz, auf den Text an die zitierte Stelle
	bibencoding = auto,
	giveninits = true,              % Vornamen werden abgekürzt
	doi=false,                      % DOI nicht anzeigen
	isbn=false,                     % ISBN nicht anzeigen
    alldates=short                  % Datum immer als DD.MM.YYYY anzeigen
]{biblatex}
\addbibresource{Inhalt/literatur.bib}
\setcounter{biburlnumpenalty}{3000}     % Umbruchgrenze für Zahlen
\setcounter{biburlucpenalty}{6000}      % Umbruchgrenze für Großbuchstaben
\setcounter{biburllcpenalty}{9000}      % Umbruchgrenze für Kleinbuchstaben
\DeclareNameAlias{default}{family-given}  % Nachname vor dem Vornamen
\AtBeginBibliography{\renewcommand{\multinamedelim}{\addslash\space
}\renewcommand{\finalnamedelim}{\multinamedelim}}  % Schrägstrich zwischen den Autorennamen
\DefineBibliographyStrings{german}{
  urlseen = {Einsichtnahme:},                      % Ändern des Titels von "besucht am"
}
\usepackage[babel,german=quotes]{csquotes}         % Deutsche Anführungszeichen + Zitate


% ---- Für Mathevorlage
\usepackage{amsmath}    % Erweiterung vom Mathe-Satz
\usepackage{amssymb}    % Lädt amsfonts und weitere Symbole
\usepackage{MnSymbol}   % Für Symbole, die in amssymb nicht enthalten sind.


% ---- Für Quellcodevorlage
\usepackage{scrhack}                    % Hack zur Verw. von listings in KOMA-Script
\usepackage{listings}                   % Darstellung von Quellcode
\usepackage{xcolor}                     % Einfache Verwendung von Farben
%% -- Eigene Farben für den Quellcode
\definecolor{JavaLila}{rgb}{0.4,0.1,0.4}
\definecolor{JavaGruen}{rgb}{0.3,0.5,0.4}
\definecolor{JavaBlau}{rgb}{0.0,0.0,1.0}
\definecolor{ABAPKeywordsBlue}{HTML}{6000ff}
\definecolor{ABAPCommentGrey}{HTML}{808080}
\definecolor{ABAPStringGreen}{HTML}{4da619}
\definecolor{PyKeywordsBlue}{HTML}{0000AC}
\definecolor{PyCommentGrey}{HTML}{808080}
\definecolor{PyStringGreen}{HTML}{008080}
% -- Farben für ABAP CDS
\definecolor{CDSString}{HTML}{FF8C00}
\definecolor{CDSKeywords}{HTML}{6000ff}
\definecolor{CDSAnnotation}{HTML}{00BFFF}
\definecolor{CDSComment}{HTML}{808080}
\definecolor{CDSFunc}{HTML}{FF0000}

% -- Default Listing-Styles

\lstset{
	% Das Paket "listings" kann kein UTF-8. Deswegen werden hier 
	% die häufigsten Zeichen definiert (ä,ö,ü,...)
	literate=%
		{á}{{\'a}}1 {é}{{\'e}}1 {í}{{\'i}}1 {ó}{{\'o}}1 {ú}{{\'u}}1
		{Á}{{\'A}}1 {É}{{\'E}}1 {Í}{{\'I}}1 {Ó}{{\'O}}1 {Ú}{{\'U}}1
		{à}{{\`a}}1 {è}{{\`e}}1 {ì}{{\`i}}1 {ò}{{\`o}}1 {ù}{{\`u}}1
		{À}{{\`A}}1 {È}{{\'E}}1 {Ì}{{\`I}}1 {Ò}{{\`O}}1 {Ù}{{\`U}}1
		{ä}{{\"a}}1 {ë}{{\"e}}1 {ï}{{\"i}}1 {ö}{{\"o}}1 {ü}{{\"u}}1
		{Ä}{{\"A}}1 {Ë}{{\"E}}1 {Ï}{{\"I}}1 {Ö}{{\"O}}1 {Ü}{{\"U}}1
		{â}{{\^a}}1 {ê}{{\^e}}1 {î}{{\^i}}1 {ô}{{\^o}}1 {û}{{\^u}}1
		{Â}{{\^A}}1 {Ê}{{\^E}}1 {Î}{{\^I}}1 {Ô}{{\^O}}1 {Û}{{\^U}}1
		{œ}{{\oe}}1 {Œ}{{\OE}}1 {æ}{{\ae}}1 {Æ}{{\AE}}1 {ß}{{\ss}}1
		{ű}{{\H{u}}}1 {Ű}{{\H{U}}}1 {ő}{{\H{o}}}1 {Ő}{{\H{O}}}1
		{ç}{{\c c}}1 {Ç}{{\c C}}1 {ø}{{\o}}1 {å}{{\r a}}1 {Å}{{\r A}}1
		{€}{{\euro}}1 {£}{{\pounds}}1 {«}{{\guillemotleft}}1
		{»}{{\guillemotright}}1 {ñ}{{\~n}}1 {Ñ}{{\~N}}1 {¿}{{?`}}1,
	breaklines=true,        % Breche lange Zeilen um 
	breakatwhitespace=true, % Wenn möglich, bei Leerzeichen umbrechen
	% Symbol für Zeilenumbruch einfügen
	prebreak=\raisebox{0ex}[0ex][0ex]{\ensuremath{\rhookswarrow}},
	postbreak=\raisebox{0ex}[0ex][0ex]{\ensuremath{\rcurvearrowse\space}},
	tabsize=4,                                 % Setze die Breite eines Tabs
	basicstyle=\ttfamily\small,                % Grundsätzlicher Schriftstyle
	columns=fixed,                             % Besseres Schriftbild
	numbers=left,                              % Nummerierung der Zeilen
	%frame=single,                             % Umrandung des Codes
	showstringspaces=false,                    % Keine Leerzeichen hervorheben
	keywordstyle=\color{blue},
	ndkeywordstyle=\bfseries\color{darkgray},
	identifierstyle=\color{black},
	commentstyle=\itshape\color{JavaGruen},   % Kommentare in eigener Farbe
	stringstyle=\color{JavaBlau},             % Strings in eigener Farbe,
	captionpos=b,                             % Bild*unter*schrift
	xleftmargin=5.0ex
}

% ---- Eigener JAVA-Style für den Quellcode
\renewcommand{\ttdefault}{pcr}               % Schriftart, welche auch fett beinhaltet
\lstdefinestyle{EigenerJavaStyle}{
	language=Java,                             % Syntax Highlighting für Java
	%frame=single,                             % Umrandung des Codes
	keywordstyle=\bfseries\color{JavaLila},    % Keywords in eigener Farbe und fett
	commentstyle=\itshape\color{JavaGruen},    % Kommentare in eigener Farbe und italic
	stringstyle=\color{JavaBlau}               % Strings in eigener Farbe
}

% ---- Eigener ABAP-Style für den Quellcode
\renewcommand{\ttdefault}{pcr}
\lstdefinestyle{EigenerABAPStyle}{
	language=[R/3 6.10]ABAP,
	morestring=[b]\|,                          % Für Pipe-Strings
	morestring=[b]\`,                          % für Backtick-Strings
	keywordstyle=\bfseries\color{ABAPKeywordsBlue},
	commentstyle=\itshape\color{ABAPCommentGrey},
	stringstyle=\color{ABAPStringGreen},
	tabsize=2,
	morekeywords={
		types,
		@data,
		as,
		lower,
		start,
		selection,
		order,
		by,
		inner,
		join,
		key,
		end,
		cast
	}
}

% ---- Eigener Python-Style für den Quellcode
\renewcommand{\ttdefault}{pcr}
\lstdefinestyle{EigenerPythonStyle}{
	language=Python,
	columns=flexible,
	keywordstyle=\bfseries\color{PyKeywordsBlue},
	commentstyle=\itshape\color{PyCommentGrey},
	stringstyle=\color{PyStringGreen}
}

%----- ABAP-CDS-View language
\lstdefinelanguage{ABAPCDS}{
	sensitive=false,
	%Keywords
	morekeywords={define,
		view,
		as,
		select,
		from,
		inner,
		join,
		on,
		key,
		case,
		when,
		then,
		else,
		end,
		true,
		false,
		cast,
		where,
		and,
		distinct,
		group,
		by,
		having,
		min,
		sum,
		max,
		count,
		avg
	},
	%Methoden
	morekeywords=[2]{
		div,
		currency\_conversion,
		dats\_days\_between,
		concat\_with\_space,
		dats\_add_days,
		dats\_is\_valid,
		dats\_add\_months,
		unit\_conversion,
		division,
		mod,
		abs,
		floor,
		ceil,
		round,
		concat,
		replace,
		substring,
		left,
		right,
		length
	},
	morecomment=[s][\color{CDSAnnotation}]{@}{:},
	morecomment=[l][\itshape\color{CDSComment}]{//},
	morecomment=[s][\itshape\color{CDSComment}]{/*}{*/},
	morestring=[b][\color{CDSString}]',
	keywordstyle=\bfseries\color{CDSKeywords},
	keywordstyle=[2]\color{CDSFunc}
}

  % Weitere Details sind ausgelagert

\usepackage{algorithm}                  % Für Algorithmen-Umgebung (ähnlich wie lstlistings Umgebung)
\usepackage{algpseudocode}              % Für Pseudocode. Füge "[noend]" hinzu, wenn du kein "endif",
                                        % etc. haben willst.

\makeatletter                           % Sorgt dafür, dass man @ in Namen verwenden kann.
                                        % Ansonsten gibt es in der nächsten Zeile einen Compilefehler.
\renewcommand{\ALG@name}{Algorithmus}   % Umbenennen von "Algorithm" im Header der Listings.
\makeatother                            % Zeichen wieder zurücksetzen
\renewcommand{\lstlistingname}{Listing} % Erlaubt das Umbenennen von "Listing" in anderen Titel.

% ---- Tabellen
\usepackage{booktabs}  % Für schönere Tabellen. Enthält neue Befehle wie \midrule
\usepackage{multirow}  % Mehrzeilige Tabellen
\usepackage{siunitx}   % Für SI Einheiten und das Ausrichten Nachkommastellen
\sisetup{locale=DE, range-phrase={~bis~}, output-decimal-marker={,}} % Damit ein Komma und kein Punkt verwendet wird.
\usepackage{xfrac} % Für siunitx Option "fraction-function=\sfrac"

% ---- Für Definitionsboxen in der Einleitung
\usepackage{amsthm}                     % Liefert die Grundlagen für Theoreme
\usepackage[framemethod=tikz]{mdframed} % Boxen für die Umrandung
%% ---- Definition für Highlight Boxen

% ---- Grundsätzliche Definition zum Style
\newtheoremstyle{defi}
  {\topsep}         % Abstand oben
  {\topsep}         % Abstand unten
  {\normalfont}     % Schrift des Bodys
  {0pt}             % Einschub der ersten Zeile
  {\bfseries}       % Darstellung von der Schrift in der Überschrift
  {:}               % Trennzeichen zwischen Überschrift und Body
  {.5em}            % Abstand nach dem Trennzeichen zum Body Text
  {\thmname{#3}}    % Name in eckigen Klammern
\theoremstyle{defi}

% ------ Definition zum Strich vor eines Texts
\newmdtheoremenv[
  hidealllines = true,       % Rahmen komplett ausblenden
  leftline = true,           % Linie links einschalten
  innertopmargin = 0pt,      % Abstand oben
  innerbottommargin = 4pt,   % Abstand unten
  innerrightmargin = 0pt,    % Abstand rechts
  linewidth = 3pt,           % Linienbreite
  linecolor = gray!40,       % Linienfarbe
]{defStrich}{Definition}     % Name der des formats "defStrich"

% ------ Definition zum Eck-Kasten um einen Text
\newmdtheoremenv[
  hidealllines = true,
  innertopmargin = 6pt,
  linecolor = gray!40,
  singleextra={              % Eck-Markierungen für die Definition
    \draw[line width=3pt,gray!50,line cap=rect] (O|-P) -- +(1cm,0pt);
    \draw[line width=3pt,gray!50,line cap=rect] (O|-P) -- +(0pt,-1cm);
    \draw[line width=3pt,gray!50,line cap=rect] (O-|P) -- +(-1cm,0pt);
    \draw[line width=3pt,gray!50,line cap=rect] (O-|P) -- +(0pt,1cm);
  }
]{defEckKasten}{Definition}  % Name der des formats "defEckKasten"  % Weitere Details sind ausgelagert

% ---- Für Todo Notes
\usepackage{todonotes}
\setlength {\marginparwidth }{2cm}      % Abstand für Todo Notizen

% ---- Zum Einbinden von PDF-Dokumenten
\usepackage{pdfpages}
